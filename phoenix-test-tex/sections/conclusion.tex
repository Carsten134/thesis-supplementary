\section{Concolusions: The Patient of Tomorrow}
Drawing from the cluster analysis and forecasting results, several characteristics emerge that define the likely VCA patient of tomorrow:

\begin{itemize}
    \item \textbf{Gender and Procedure.} Gender remains the strongest axis of separation. Female patients, particularly those undergoing uterus transplantation, are projected to represent an increasing share of future VCA recipients.
    \item \textbf{Age and Medical Profile.} The female cohort is younger on average (ages 20--43) with moderate BMI values and shorter waiting times compared to male cohorts. Especially young women with blood type O and A receiving uterus transplants form a distinct subgroup.
    \item \textbf{Male Cohort Outlook.} Male patients are dominated by face and upper limb transplants, but these groups display stagnating or declining trends. The reduction in upper limb procedures may be linked to advances in prosthetic technology, while face transplants continue but are increasingly complemented by regenerative alternatives such as tissue engineering and 3D bioprinting.
    \item \textbf{Geographic Concentration.} Uterus transplants show a clear concentration in UNOS Regions 4 and 10, with recent activity also in Region 3. Male cohorts, in contrast, remain more geographically dispersed across Regions 1, 9, and 11.
\end{itemize}

\paragraph{Conclusion.}
The ``patient of tomorrow'' is therefore most likely a \textit{young female undergoing a uterus transplant}. In contrast, traditional strongholds of male VCA (face and upper limb) may diminish in prominence, either due to stagnation in clinical uptake or competition from technological innovations in prosthetics and reconstructive medicine. Nevertheless, given the small sample sizes and limited predictive power of the models, these conclusions must be interpreted with caution.