\section{Introduction}
\cite{jentschTestingEqualitySpectral2015} motivate their use of $L_2$ tests by the inconsistency of raw periodogram tests. Frankly, periodogram tests have been well known for their inconsistent estimate of the spectral density and using them for teststatistics seems to prove very unfavourable at least looking towards asymptotic behaviour. 

However the problem induced with $L_2$ teststatistics, defined in a more general way in \cite{eichlerTestingNonparametricSemiparametric2008}, is the slow convergence twoards the very favourable and easy to use asymptotic distribution, justifying the use of randomization techniques and giving way to the paper \cite{jentschTestingEqualitySpectral2015} and my Master's thesis. In this context it was suprising to find the work of \cite{scacciaTestingAxialSymmetry2005} in which they propose a test for axial symmetry on lattice processes using the periodogram. Two observations stood out from their work, particularly regarding their third teststatistic $T_3$:
\begin{enumerate}
    \item The speed of convergence to the normal distribution, needing merely a $11\times 11$ lattice to be near enough to the normal distribution to properly hold it's size
    \item Outstanding results from powersimulation regarding $T_3$
\end{enumerate} 
Convinced by their results, I attempted to apply $T_3$ to the equality of spectraldensities testingproblem. This document serves as a quick and consise presentation of the results of my efforts.
